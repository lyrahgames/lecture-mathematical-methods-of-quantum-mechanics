\documentclass[8pt,twocolumn]{article}
\usepackage{extsizes}
\usepackage[a4paper,top=15mm,bottom=15mm,right=20mm,left=20mm]{geometry}

\usepackage[utf8]{inputenc}
\usepackage[T1]{fontenc}
\usepackage[english]{babel}
\usepackage[light,sfdefault]{roboto}
\usepackage{dsfont}

\usepackage{amsmath, mathtools}
\usepackage[mathscr]{euscript}

\begin{document}
  \section{Preliminaries} % (fold)
  \label{sec:preliminaries}
    \subsection{Metric Spaces} % (fold)
    \label{sub:metric_spaces}
      \begin{itemize}
        \item Definition: Metric and Metric Space
        \item Definition: Induced Metric
        \item Examples: $d_1$, $d_\infty$, $d_2$ for $\mathds{R}^n$ and $C(I)$
        \item Definition: Convergent Sequences and Limits
        \item Lemma: Limit is Unique
        \item Definition: Cauchy Sequence
        \item Proposition: Convergent Sequence is Cauchy Sequence
        \item Remark: $d_1$ restricted to $\mathds{Q}$ is not dense.
        \item Definition: Complete Metric Spaces
        \item Definition: Dense Subsets
        \item Definition: Continuity of Functions
        \item Proposition: Characterization of Continuity
        \item Definition: Isometry and Isometric Metric Spaces
        \item Theorem: Unique Completion of Isometric Dense Metric Subspaces
        \item Definition: Open and Closed Ball, Open Sets, Neighborhood and Interior Points
        \item Definition: Set of Limit Points, Closed Sets and Closures
        \item Lemma: Characterization of Limit Points and Closed Sets
        \item Lemma: Closed Subspaces of Complete Spaces are Complete
        \item Remark: Characterization of Density of Subspace
        \item Theorem: Rules for Open and Closed Sets
        \item Theorem: Characterization of Continuous Functions
        \item Theorem: Product Metric
      \end{itemize}
    % subsection metric_spaces (end)

    \subsection{Normed Vector Spaces} % (fold)
    \label{sub:normed_vector_spaces}
      \begin{itemize}
        \item Definition: Linear Operator
        \item Definition: Norm, Seminorm and Normed Spaces
        \item Remark: Normed Spaces Induce Metric Spaces
        \item Example: Minkowski-Norm for $\mathds{R}^n$ and $C(I)$
        \item Lemma: Young's Inequality
        \item Theorem: Hölder's Inequalities
        \item Theorem: Minkowski's Inequalities
        \item Theorem: Second Version of Triangle Inequality
        \item Definition: Equivalent Norms
        \item Remark: Finite Dimensional Norms are Equivalent, Indistinguishability
        \item Remark: Product Metric
        \item Proposition: $+$ and $\cdot$ of Normed Spaces are Continuous
        \item Definition: Bound linear Transformations
        \item Lemma: Normed Space of Bounded Linear Operators and Norm Characterizations
        \item Theorem: Characterization of Bounded Linear Operators
        \item Definition: Banach Space
        \item Theorem: Completion of Normed Vector Spaces
        \item Remark: $L^p(I)$ Spaces
        \item Theorem: BLT-Theorem
        \item Remark: Lebesgue-Integral as Extension of $C_c(\mathds{R}^n)$
        \item Theorem: Characterization of Completeness
      \end{itemize}
    % subsection normed_vector_spaces (end)
  % section preliminaries (end)

  \section{Measure Theory} % (fold)
  \label{sec:measure_and_integration_theory}
    \subsection{Measure Theory} % (fold)
    \label{sub:measure_theory}
      \begin{itemize}
        \item Measurable Spaces and Measure Spaces
          \begin{itemize}
            \item Definition: Algebra, $\sigma$-Algebra, Measurable Space, Measurable Sets
            \item Example: Trivial $\sigma$-Algebras
            \item Lemma: Properties of Algebras and $\sigma$-Algebras
            \item Definition: Generator of $\sigma$-Algebra
            \item Remark: Existence and Uniqueness of Smallest $\sigma$-Algebra
            \item Definition: Borel-$\sigma$-Algebra
            \item Lemma: Monotonicity of Generator
            \item Example: Generators of Borel-$\sigma$-Algebra
            \item Example, Proposition, Lemma, Corollary: Generator of Product Borel-$\sigma$-Algebra
            \item Definition: Product $\sigma$-Algebra
            \item Proposition, Example: Elementary Family Induces Algebra
            \item Definition: Measure, Measure Spaces, finite and $\sigma$-finite
            \item Example: Dirac Point Measure
            \item Theorem: Monotonicity, Subadditivity, Continuity from Below and Above
            \item Definition: Null Set, Almost Everywhere Holding Statements
            \item Definition: Complete Measures
            \item Theorem, Definition: Completion of Measure Space
          \end{itemize}
        \item Lebesgue-Measure
          \begin{itemize}
            \item Definition: Lebesgue Outer Measure
            \item Remark: Lebesgue Outer Measure is well defined
            \item Theorem: Restriction of Lebesgue Outer Measure to Borel-$\sigma$-Algebra is a Measure
            \item Definition: Lebesgue Measure and Lebesgue-measurable Sets
            \item Theorem: Measurable Sets
            \item Theorem: Characterization Lebesgue-Measurability
          \end{itemize}
        \item Construction of Measures from Outer Measures
          \begin{itemize}
            \item Definition: Outer Measure
            \item Lemma: Construction of Outer Measure
            \item Example: Lebesgue Outer Measure is Outer Measure
            \item Definition, Remark: Measurable Set with Respect to Outer Measure
            \item Theorem: Caratheodory, Existence of Measures
            \item Lemma: Open Intervals are outer-measurable
            \item Theorem: Characterization of Lebesgue-Measurability
          \end{itemize}
        \item Uniqueness of Measures and Monotone Classes
          \begin{itemize}
            \item Definition: Monotone Class and Generator
            \item Theorem: Monotone Class Theorem
            \item Theorem: Uniqueness of Measures
          \end{itemize}
      \end{itemize}
    % subsection measure_theory (end)

    \subsection{Integration Theory} % (fold)
    \label{sub:integration_theory}
      \begin{itemize}
        \item Integration
          \begin{itemize}
            \item Measurable Functions
            \item Lemma: Characterization, Composition, Continuity
            \item Proposition: Measurable Functions on Product Spaces
            \item Corollary: Coordinates, Sum and Product are Measurable
            \item Theorem: sup, inf, limsup, liminf, lim are Measurable
            \item Definition: Simple Functions
            \item Lemma: Simple Function Formulation with Linear Combination, Characterization Measurability
            \item Theorem: Monotone Convergence of Simple Functions to Measurable Functions
            \item Definition: Integral of Simple Functions
            \item Proposition: Measure, Linearity and Monotonicity of Positive Simple Function Integrals
            \item Definition: Lebesgue Integral for Non-negative Functions
            \item Remark: Properties of Lebesgue Integral
            \item Theorem: Monotone Convergence Theorem by Lebesgue
            \item Corollary: Linearity, Interchange of Sum and Integral
            \item Theorem: Lemma by Fatou
            \item Definition: Lebesgue Integral
            \item Proposition, Definition: $L^1$ Space
            \item Theorem: Dominated Convergence
            \item Theorem: Parameter dependent Integrals
          \end{itemize}
        \item Product Measures
          \begin{itemize}
            \item Proposition: Section Property
            \item Theorem: Product Measure
            \item Theorem: Fubini-Tonelli
          \end{itemize}
        \item $L^p$ Spaces
          \begin{itemize}
            \item Definition: $L^p$ Spaces
            \item Lemma: $L^p$ is a Vector Space
            \item Theorem: Hölder's Inequality
            \item Theorem: Minkowski's Inequality
            \item Theorem: $L^p$-Norm Implies almost everywhere Pointwise Convergence of Subsequence
            \item Corollary: Cauchy Sequence
            \item Theorem: Set of Simple Functions is Dense in $L^p$
            \item Corollary: $L^p$ has countable dense subset, $C_c(\mathds{R}^n)$ is Dense in $L^p(\mathds{R}^n)$
            \item Definition: Essential Supremum and $L^\infty$
            \item Theorem: Properties of $L^\infty$
          \end{itemize}
      \end{itemize}
    % subsection integration_theory (end)
  % section measure_and_integration_theory (end)

  \section{Hilbert Spaces} % (fold)
  \label{sec:hilbert_spaces}
    \subsection{Sesquilinearforms} % (fold)
    \label{sub:sesquilinearforms}
      \begin{itemize}
        \item Definition: Sesquilinearform, Symmetry, Non-negativity, Positive Definite, Bilinear Form
        \item Lemma: $s(0,x) = s(x,0) = 0$
        \item Proposition: Parallelogram Identity
        \item Proposition: Polarization
        \item Corollary: Positive Sesquilinearforms are symmetric
      \end{itemize}
    % subsection sesquilinearforms (end)

    \subsection{Vector Spaces with Semi-Inner Products} % (fold)
    \label{sub:vector_spaces_with_semi_inner_products}
      \begin{itemize}
        \item Definition: Semi-Inner Product and Inner Product
        \item Lemma: Antilinearity in First Argument
        \item Remark: Characterization Semi-Inner Product and Inner Product
        \item Examples
        \item Proposition: Cauchy-Schwarz-Bunyakowski
        \item Definition: Semi-Norm
        \item Lemma: Semi-Norm Properties
        \item Proposition: Induced Semi-Norm and Norm
        \item Definition: Orthogonal and Orthogonal Complement
        \item Definition: Linear Span
        \item Lemma: Algebraic Properties of Orthogonal Complement
        \item Definition: Orthonormal System
        \item Proposition: Pythagoras
        \item Definition: General Sum
        \item Lemma: Sum over uncountable elements
        \item Theorem: Bessel
        \item Proposition: Gram-Schmidt Algorithm
      \end{itemize}
    % subsection vector_spaces_with_semi_inner_products (end)

    \subsection{Hilbert Spaces} % (fold)
    \label{sub:hilbert_spaces}
      \begin{itemize}
        \item Lemma: Induced Metric
        \item Proposition: Norm and Inner Product are Continuous
        \item Definition: Hilbert Space
        \item Lemma: Continuity Properties
        \item Lemma: Characterization of Norm and Operator Norm through Inner Product
        \item Proposition: $l^2(\mathds{N})$
        \item Theorem: Approximation Theorem
        \item Theorem: Projection Theorem
        \item Remark: Direct Sum
        \item Lemma: Completeness Properties
        \item Definition: Projection
        \item Lemma: Projection Properties
        \item Definition: Orthogonal Projection
        \item Lemma: Orthogonal Projections are Bounded
        \item Proposition: Orthogonal Projection for Closed Subspace
        \item Definition: Orthonormal Basis
        \item Proposition: Orthonormal Basis in $l^2(\mathds{N})$
        \item Definition: General Sum Convergence in Normed Space
        \item Theorem: Coefficient Representation
        \item Lemma: Characterization Orthonormal Basis
        \item Corollary: Physics Identity
        \item Theorem: Every Hilbert Space has an Orthonormal Basis
        \item Proposition: Orthogonal Projection Coefficients for Closed Subspace
      \end{itemize}
    % subsection hilbert_spaces (end)

    \subsection{Separable Hilber Spaces} % (fold)
    \label{sub:separable_hilber_spaces}
      \begin{itemize}
        \item Definition: Dense, Separable
        \item Definition: Unitary Mapping, Unitarily Equivalence
        \item Lemma: Characterization Unitarity
        \item Proposition: Characterization Separability
        \item Definition: Completion
        \item Theorem: Existence and Uniqueness of Completion
      \end{itemize}
    % subsection separable_hilber_spaces (end)

    \subsection{Riesz's Theorem} % (fold)
    \label{sub:riesz_s_theorem}
      \begin{itemize}
        \item Lemma: Continuous Linear Mappings $l_v$
        \item Theorem: Riesz's Theorem
        \item Corollary: Unique Bounded Linear Transformation for Sesquilinearforms
      \end{itemize}
    % subsection riesz_s_theorem (end)

    \subsection{Hilbert Adjoint for Bounded Operators} % (fold)
    \label{sub:hilbert_adjoint_for_bounded_operators}
      \begin{itemize}
        \item Lemma: Inner Product Identity
        \item Lemma: Existence of unique Hilbert Adjoint
        \item Lemma: Properties of Adjoint
        \item Definition: Selfadjointness and Normality
        \item Lemma: Selfadjoint Operators are normal
        \item Lemma: Characterization of Normal Operators
        \item Lemma: Characterization of Selfadjointness
        \item Corollary: Zero Identity for Selfadjoint Operators
        \item Definition: Unitary Operator
        \item Lemma: Characterization Unitary Operator
        \item Lemma: Characterization Orthogonal Projection
      \end{itemize}
    % subsection hilbert_adjoint_for_bounded_operators (end)

    \subsection{Construction of Hilbert Spaces} % (fold)
    \label{sub:construction_of_hilbert_spaces}
      \begin{itemize}
        \item Proposition: Direct Sums of Hilbert Spaces
        \item Proposition: Hilbert Space of Vector Valued Functions
        \item Definition, Lemma: Tensor Product of Hilbert Spaces
        \item Proposition: Orthonormal Basis of Tensor Product
        \item Example: General Fock Space, Symmetric Fock Space, Antisymmetric Fock Space
        \item Theorem: Tensor Product of $L^2$ Spaces
        \item Examples
      \end{itemize}
    % subsection construction_of_hilbert_spaces (end)
  % section hilbert_spaces (end)

  \section{Fourier Analysis} % (fold)
  \label{sec:fourier_analysis}
    \subsection{Existence of Test Functions} % (fold)
    \label{sub:existence_of_test_functions}
      \begin{itemize}
        \item Definition: Support
        \item Lemma: Existence of $C_C^\infty(\mathds{R}^n)$ Function
      \end{itemize}
    % subsection existence_of_test_functions (end)

    \subsection{$L^p$ Spaces in $\mathds{R}^n$} % (fold)
      \begin{itemize}
        \item Definition: Convolution
        \item Definition: $L_\mathrm{loc}^p(U)$
        \item Lemma: Convolution is continuous for $C_c$ and $L_\mathrm{loc}$
        \item Theorem: Young Inequality
        \item Proposition: Convolution Properties
        \item Proposition: Convolution Preserves Differentiability
        \item Theorem: $\phi_t$ Test Functions
        \item Corollary: $C^\infty_c$ is dense in $L^p$ and $C_0$
        \item Theorem: Zero Identity for $C^\infty_c$ Test Functions
      \end{itemize}
    % subsection subsection_name (end)

    \subsection{The Fourier Transform} % (fold)
    \label{sub:the_fourier_transform}
      \begin{itemize}
        \item Definition: Fourier Transformation and Inverse
        \item Lemma: Fourier Transform Properties
        \item Lemma: Parseval (Interchange of Convolution)
        \item Theorem: Convolution and Fourier Transform
        \item Lemma: Fourier Transform of Gaussians
      \end{itemize}
    % subsection the_fourier_transform (end)

    \subsection{Schwartz Space} % (fold)
    \label{sub:schwartz_space}
      \begin{itemize}
        \item Definition: Schwartz Space $\mathscr{S}$
        \item Lemma: $C_c^\infty$ is Dense in $\mathscr{S}$
        \item Lemma: Fourier Transform and Inverse are Linear and Interchange Multiplication with Differentiation
        \item Theorem: Fourier Transform is Bijection on $\mathscr{S}$
        \item Definition: Fourier Transform on $L^2$
        \item Theorem: Definitions of Fourier Transform agree on $L^1$ and $L^2$
        \item Theorem: Fourier Transform is Bijection on $L^2$
      \end{itemize}
    % subsection schwartz_space (end)

    \subsection{Dynamics of Free Schrödinger Equation} % (fold)
      \begin{itemize}
        \item Theorem: Solutions $\psi_t$ of Schrödinger Equation
        \item Lemma: Reformulation of Solution
        \item Corollary: Decay of Wave Packet
        \item Corollary: Solution can be Splitted into Sum
        \item Theorem: Norm Equality
      \end{itemize}
    % subsection dynamics_of_free_schrödinger_equation (end)

    \subsection{Weak Derivatives} % (fold)
    \label{sub:weak_derivatives}
      \begin{itemize}
        \item Remark: Motivation
        \item Definition: Weak Partial Derivative
        \item Remark: $L^p_\mathrm{loc}(U)\subset L^1_\mathrm{loc}(U)$
        \item Lemma: Uniqueness of Weak Derivatives
      \end{itemize}
    % subsection weak_derivatives (end)

    \subsection{Sobolev Spaces} % (fold)
    \label{sub:sobolev_spaces}
      \begin{itemize}
        \item Definition: Sobolev Space and Sobolev Norm
        \item Theorem: Sobolev Spaces are Banach Spaces
        \item Lemma: Characterization $H^n$ and Application of Fourier Transform
      \end{itemize}
    % subsection sobolev_spaces (end)
  % section fourier_analysis (end)

  \section{Unbounded Operators} % (fold)
  \label{sec:unbounded_operators}
    \begin{itemize}
      \item Theorem: Closed Graph Theorem
    \end{itemize}

    \subsection{Definitions and Basic Properties} % (fold)
    \label{sub:definitions_and_basic_properties}
      \begin{itemize}
        \item Definition: General Linear Operator
        \item Definition: Graph of Linear Operator
        \item Lemma: Equivalences for Graph
        \item Definition: Extension of a Linear Operator
        \item Theorem: Characterization of Extension
        \item Definition: Addition and Composition
      \end{itemize}
    % subsection definitions_and_basic_properties (end)

    \subsection{Closed Operators} % (fold)
    \label{sub:closed_operators}
      \begin{itemize}
        \item Definition: Closed Linear Operator
        \item Theorem: Characterization of Closed Operator
        \item Corollary: Bounded Operators are Closed Operators
        \item Definition: Closable Operators
        \item Theorem: Existence of Unique Smallest Closed Extension
        \item Theorem: $\overline{\Gamma(T)} = \Gamma(\overline{T})$
        \item Theorem: Sequence Criterion
        \item Theorem: BLT Theorem
        \item Lemma: Inverse is closed iff Operator is closed
        \item Lemma: Addition and Right Composition Preserve Closedness
        \item Definition: Graph Norm
        \item Theorem: Characterization Closedness by Completeness
      \end{itemize}
    % subsection closed_operators (end)

    \subsection{Spectral Theory} % (fold)
    \label{sub:spectral_theory}
      \begin{itemize}
        \item Definition: Resolvent Set, Resolvent and Spectrum
        \item Definition: Eigenvalue, Eigenvector and Point Spectrum
        \item Lemma: Point Spectrum is Subset of Spectrum
        \item Lemma: Properties of Resolvent
        \item Remark: Closedness implies Boundness of Inverse
        \item Remark: Operator with Nonempty Spectrum must be Closed
        \item Lemma: Resolvent Identities
        \item Theorem: Neumann Series
        \item Lemma: Disk Contained in Resolvent
        \item Theorem: Resolvent Set is Open and Spectrum is Closed, Resolvent-valued function is Analytic
        \item Theorem: Spectrum for Closed Linear Operators
      \end{itemize}
    % subsection spectral_theory (end)

    \subsection{Operators in Hilbert Spaces} % (fold)
    \label{sub:operators_in_hilbert_spaces}
      \begin{itemize}
        \item Definition: Multiplication Operator
        \item Lemma: $M_fM_g \subset M_{fg}$
        \item Definition: Measure Support and Essential Supremum
        \item Proposition: Properties of Multiplication Operator
        \item Example: Laplacian
      \end{itemize}
    % subsection operators_in_hilbert_spaces (end)

    \subsection{Adjoint Operators} % (fold)
    \label{sub:selfadjoint_operators}
      \begin{itemize}
        \item Definition: Adjoint Operator
        \item Remark: Characterizations for Domain of Definition
        \item Remark: Inner Product Identity and $V$ function
        \item Lemma: Graph of Dense Operator
        \item Lemma: Closability Properties of Dense Operator
        \item Proposition: Kernel of Adjoint is equal to Orthogonal Complement of Image of Dense Operator
        \item Lemma: Adjoint of Inverse of Resolvent
        \item Lemma: Addition and Composition Properties
        \item Lemma: Adjoint of Multiplication Operator
      \end{itemize}
    % subsection selfadjoint_operators (end)

    \subsection{Selfadjoint and Symmetric Operators} % (fold)
    \label{sub:selfadjoint_and_symmetric_operators}
      \begin{itemize}
        \item Definition: Selfadjoint Operator
        \item Remark: Selfadjointness implies Closedness
        \item Example: Laplacian on $H^2$ is selfadjoint
        \item Theorem: Properties of Selfadjoint Operators
        \item Definition: Symmetric Operator
        \item Remark: Selfadjoint Operators are Symmetric
        \item Lemma: Properties of Symmetric Operators
        \item Theorem: Basic Criterion for Selfadjointness
        \item Definition: Essentially Selfadjoint Operators
        \item Example: Laplacian restricted to Schwartz Space is essentially selfadjoint
      \end{itemize}
    % subsection selfadjoint_and_symmetric_operators (end)

    \subsection{Kato Rellich Theorem} % (fold)
    \label{sub:kato_rellich_theorem}
      \begin{itemize}
        \item Definition: $T$-Boundness and Relative Bound
        \item Theorem: Kato-Rellich
        \item Lemma: Boundness by Laplacian
        \item Corollary: Laplacian summed with sum over Functions is selfadjoint
        \item Example
      \end{itemize}
    % subsection kato_rellich_theorem (end)
  % section unbounded_operators (end)

  \section{Spectral Theorem} % (fold)
  \label{sec:spectral_theorem}
    \begin{itemize}
      \item Theorem: Spectral Theorem - Multiplication Operator Form
      \item Theorem: Spectral Theorem - Multiplication Operator Form - Separable Case
      \item Remark: Physics Notation
    \end{itemize}
  % section spectral_theorem (end)
\end{document}